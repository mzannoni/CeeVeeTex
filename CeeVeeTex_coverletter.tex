% Massimo Zannoni - Curriculum Vitae
%
% This work is licensed under a Creative Commons Attribution-ShareAlike 4.0 International License (CC BY-SA 4.0)

\documentclass[a4paper]{letter}
%\usepackage{graphicx}
\usepackage[top=2.4cm, bottom=2.4cm, left=2cm, right=2cm]{geometry}
\usepackage[utf8]{inputenc}
\usepackage{hyperref}
\usepackage[english]{babel}

%% set the font family
\usepackage[condensed,math]{kurier}
%% set the font encoding
\usepackage[T1]{fontenc}
%% details about the font
%% https://ctan.org/tex-archive/fonts/kurier/

% Personal details
\def\name{Name Surname}
\def\email{yourmail@goeshere.com}
\def\telephone{+00 11 22 33 44}
\def\nationality{Italian}
\def\dateofbirth{31.12.2020}
\def\gender{Male}
\def\addrs{Mystreet, 1 \\ 0123 MyCity \\ MyCountry}

% Addressee details
\def\company{Target Company}
\def\recruiter{The Recruiter}
\def\companyaddress{Somestreet, 1 \newline 0000 City \newline Country}

% Application details
\def\lettersubject{Subject: Application for this position}

% Edit PDF file properties
\hypersetup{
  pdfauthor = {\name},
  pdfcreator = {\name},
  pdfkeywords = {personal_skill_01, personal_skill_02, personal_interest_01, etc},
  pdftitle = {\name ~- Cover Letter},
  pdfsubject = {Cover Letter},
  % next line to avoid links from being blue and underlined
  hidelinks
}

% Letter commands
\signature{\name}
\address{\addrs}

% the following 2 params affect the layout of the closing:
%
% this one keeps the closing all to the left
% \longindentation=0pt
% this one sets the vertical space between salutation and signature
% \medskipamount=0.2\parskip

\begin{document}
  \begin{letter}{\recruiter \\ \company \\ \companyaddress \\ ~ \\ \textbf{\lettersubject}}
  % or if no addressee is needed:
  % \begin{letter}{\textbf{\lettersubject}}
  % and without even a subject line:
  % \begin{letter}{}

  \opening{To the kind attention of \recruiter ~in \company}

  Here you would have to write your cover letter. It's not the scope of this template to give any indication on the content nor the style.\\
  I'm also not the right person to give such kind of hints, I believe.\\
  You will find around plenty of guides, howtos and tutorials on what to write in a cover letter, how to write it, who to address and so on.\\
  The content and the style are also highly dependent on the specific field.\\

  \closing{Sincerely,}

  \end{letter}
\end{document}
